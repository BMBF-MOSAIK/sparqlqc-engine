\documentclass{beamer}

%\usepackage[backend=bibtex,citestyle=authoryear]{biblatex}
%\addbibresource{Entailment}
%\addbibresource{Fragments}
%\addbibresource{Benchmarks}

\usepackage{amsmath}
\usepackage{stmaryrd}
\usepackage{listings}
% Language Definitions for Turtle
\definecolor{olivegreen}{rgb}{0.2,0.8,0.5}
\definecolor{grey}{rgb}{0.5,0.5,0.5}

%\lstdefinelanguage{ttl}{
%sensitive=true,
%morecomment=[l][\color{grey}]{@},
%morecomment=[l][\color{olivegreen}]{\#},
%morestring=[b][\color{blue}]\",
%}

% Language Definitions for RDF/Turtle
\lstdefinelanguage{ttl} {
 morekeywords={@prefix,@base,@forSome,@forAll,@keywords},
 sensitive=true,
% morecomment=[l][\color{grey}]{@},
% morecomment=[l][\color{olivegreen}]{\#},
 morestring=[b][\color{blue}]\",
}

% Language Definitions for SPARQL
\lstdefinelanguage{sparql}{
%morecomment=[l][\color{olivegreen}]{\#},
morestring=[b][\color{blue}]\",
morekeywords={SELECT,CONSTRUCT,DESCRIBE,ASK,WHERE,FROM,NAMED,PREFIX,BASE,OPTIONAL,FILTER,GRAPH,LIMIT,OFFSET,SERVICE,UNION,EXISTS,NOT,BINDINGS,MINUS,a},
sensitive=true
}

\lstset{
    basicstyle=\ttfamily\scriptsize,
%    basicstyle=\ttfamily\tiny,
    keywordstyle=\bfseries,
    numbers=none,
    extendedchars=true,
    showlines=true,
    upquote=true,
    showstringspaces=false,
    xleftmargin=5mm
}

% left join operator symbol
\def\ojoin{\setbox0=\hbox{$\bowtie$}%
  \rule[-.02ex]{.25em}{.4pt}\llap{\rule[\ht0]{.25em}{.4pt}}}
\def\leftouterjoin{\mathbin{\ojoin\mkern-5.8mu\bowtie}}
\def\rightouterjoin{\mathbin{\bowtie\mkern-5.8mu\ojoin}}
\def\fullouterjoin{\mathbin{\ojoin\mkern-5.8mu\bowtie\mkern-5.8mu\ojoin}}


\setbeamertemplate{section page}
{
    \begin{centering}
    \begin{beamercolorbox}[sep=12pt,center]{part title}
    \usebeamerfont{section title}\insertsection\par
    \end{beamercolorbox}
    \end{centering}
}



\title{Solving SPARQL Query Containment}
\subtitle{using (Sub-) Graph Isomorphisms}
\author[shortname]{Ren\'{e} Schubotz \textsuperscript{1}}
\institute[shortinst]{\textsuperscript{1} German Research Center for Artificial Intelligence}





%\usetheme{lucid}
\begin{document}
\frame {
	\titlepage
}

\begin{frame}
\frametitle{Overview} 
\tableofcontents
\end{frame}

% ========================================================================================
\section{Resource Description Framework}
\frame{\sectionpage}

\frame {
\frametitle{Resource Description Framework}
\framesubtitle{Triples, Graphs and Datasets}
 Let $\mathbf{I}$, $\mathbf{L}$ and $\mathbf{B}$ be pairwise disjoint infinite sets of \texttt{IRI}s, literals and blank nodes.
 \vspace{1em}
 
 The infinite set of all \textit{RDF triples} is $\mathcal{T} = \mathbf{IB} \times \mathbf{I} \times \mathbf{IBL}$.
 \vspace{1em}
 
 An \textit{RDF graph} $G \subset \mathcal{T}$ is a finite set of RDF triples.
 \vspace{1em}
 
 An \textit{RDF dataset} $\mathcal{D}$ is a set $\{ G_{0}, \langle u_1, G_1 \rangle, \dots , \langle u_n, G_n \rangle \}$, where $G_0$ is the 		default graph, and $G_1, \dots , G_n$ are named graphs with names $u_1, \dots , u_n \in \mathbf{I}$.
}

\begin{frame}[fragile]
\frametitle{Resource Description Framework}
\framesubtitle{Example: RDF graph}
\begin{lstlisting}[language=ttl]
@prefix foaf:  <http://xmlns.com/foaf/0.1/> .

_:a  foaf:name   "Johnny Lee Outlaw" .
_:a  foaf:mbox   <mailto:jlow@example.com> .
_:b  foaf:name   "Peter Goodguy" .
_:b  foaf:mbox   <mailto:peter@example.org> .
_:c  foaf:mbox   <mailto:carol@example.org> .
\end{lstlisting}
\begin{figure}
\includegraphics[width=0.45\textwidth]{graph.pdf}
\end{figure}
\end{frame}


\begin{frame}[fragile]
\frametitle{Resource Description Framework}
\framesubtitle{Simple Interpretations of RDF graphs}

A \textbf{simple interpretation} $\mathcal{I}$ is a tuple $\mathcal{I} = (\mathbb{R}, \mathbb{P}, \circ_\text{Int}, \circ_\text{Ext}, \circ_\text{L} )$ such that
\begin{itemize}
\item $\mathbb{R}$ is a non-empty set of resources, called the domain or also universe of discourse of $\mathcal{I}$,
%\item $\text{LV} \subseteq \text{R}$ is the set of literal values, that contains at least all untyped literals from $V$,
\item $\mathbb{P}$ is the set of all properties\footnote{Not necessarily disjoint from or a subset of $\mathbb{R}$.},
\item ${\circ}_\mathbf{Ext}: \mathbb{P} \rightarrow 2^{\mathbb{R} \times \mathbb{R}}$ is a mapping that assigns a binary relational extension to each property,
\item ${\circ}_\mathbf{Int}: \mathbf{I} \rightarrow ( \mathbb{R} \cup \mathbb{P} )$ is the interpretation mapping that assigns a resource or property to a given \texttt{IRI},
\item ${\circ}_\mathbf{L}: \mathbf{L} \rightarrow \mathbb{R}$ is a partial mapping from literals into the set of literal values\footnote{The set of literal values is a subset of the set of resources $\mathbb{R}$.}
\end{itemize}
\end{frame}


\begin{frame}[fragile]
\frametitle{Resource Description Framework}
\framesubtitle{s-Models and s-Satisfiability of RDF graphs}

A simple interpretation $\mathcal{I}$ is a \textit{model} of an RDF graph $G$ iff
\begin{itemize}
\item If $e \in \mathbf{L}$, then $\mathcal{I}(e) =  {\circ}_\mathbf{L}(e)$.
\item If $e \in \mathbf{IB}$, then $\mathcal{I}(e) =  {\circ}_\mathbf{Int}(e)$.
\item If $e = (s, p, o) \in \mathbf{IB} \times \mathbf{I}  \times \mathbf{IBL}$ , then $\mathcal{I}(e) = \top$ iff $\mathcal{I}(p) \in \mathbb{P}$ and $( \mathcal{I}(s), \mathcal{I}(o) ) \in {\circ}_\mathbf{Ext}(\mathcal{I}(p))$.
\item If $e$ is an RDF graph, then $\mathcal{I}(e) = \bot$ iff $\mathcal{I}(e') = \bot$ for some RDF triple $e' \in e$, otherwise $\mathcal{I}(e) = \top$.
\end{itemize}

\vspace{2em}

An RDF graph $G$ is (simply) \textit{satisfiable} if a simple interpretation $\mathcal{I}$ exists with $\mathcal{I}(G) = \top$, otherwise $G$ is (simply) \textit{unsatisfiable}.

\end{frame}



\begin{frame}[fragile]
\frametitle{Resource Description Framework}
\framesubtitle{s-Entailment, s-Equivalence and Leanness of RDF graphs}

We say that the RDF graph $G$ (simply) \textit{entails} the RDF graph $H$ ($G \models H$) if and only if every model $\mathcal{I}$ which satisfies $G$ also satisfies $H$. 

\vspace{2em}

Two RDF graphs $G$ and $H$ are \textit{equivalent} ($G \equiv H$) if and only if $G \models H$ and $H \models G$.

\vspace{2em}

An RDF graph $G$ is considered \textit{lean} if and only if there does not exist a proper subgraph $G' \subset G$ such that $G' \models G$.
\end{frame}



% ========================================================================================
\section{More RDF Entailment Regimes}
\frame{\sectionpage}

\frame{
\frametitle{More RDF Entailment Regimes}
\framesubtitle{Literals and Datatypes}

An \textit{RDF datatype} $d$ is defined as a 4-tuple $(\mathcal{L}_{d}, \mathcal{V}_{d}, \mu_{d}, \mathbf{D}_{d})$, where $\mathcal{L}_{d}$ is its lexical space, $\mathcal{V}_{d}$ is its value space, $\mu_{d}$ gives the datatype's lexical-to-value mapping, and $\mathbf{D}_{d}$ is the set of its datatype \texttt{IRI}s. 
\vspace{2em}

The subset $\mathbf{L}$ of RDF terms is called \textit{RDF literals} and characterized as 
$\mathbf{L} = \mathbf{L}_{t} \cup \mathbf{L}_{l}$ where $\mathcal{L}$ is the infinite set of \textbf{lexical forms} and $\mathbf{L}_{t} = ( \mathcal{L} \times \binom{\mathbf{D}}{1} )$ with $\mathbf{D}$ is the set of RDF datatype \texttt{IRI}s, and $\mathbf{L}_{l} = ( \mathcal{L} \times \{ \mathtt{rdf:langString} \} \times \mathcal{T})$ with $\mathcal{T}$ is the is the set of non-empty and well-formed \textbf{language tags}.
}

\frame{
\frametitle{More RDF Entailment Regimes}
\framesubtitle{D-Interpretation, D-Satisfiability and D-Entailment}

Let $\mathbf{D}$ be the set of datatype \texttt{IRI}s. A \textit{(simple) D-interpretation} $\mathcal{I}_{\mathbf{D}}$ is a s-interpretation satisfying
\footnotesize
\begin{itemize}
\item ${\circ}_\mathbf{L}(l) = \mu_{\mathcal{I}(\text{ddd})}(\text{sss}) = \mu_{d}(\text{sss}) \iff l = (\text{sss}, \text{ddd}) \in \mathbf{L}_{t}$,
\item ${\circ}_\mathbf{L}(l) = ( \text{sss}, \text{ttt} ) \iff l = (\text{sss}, \mathtt{rdf:langString}, \text{ttt}) \in \mathbf{L}_{l}$,
\item ${\circ}_\mathbf{L}(l) = \bot$, otherwise
\end{itemize}
\normalsize
under the condition that $\text{sss} \in \mathcal{L}$, $\text{ttt} \in \mathcal{T}$, $\mathtt{rdf:langString} \in \mathbf{D}$ and $\text{ddd} \in \mathbf{D}$.
\vspace{2em}

An RDF graph $G$ is (simply) \textit{D-satisfiable} if a D-interpretation $\mathcal{I}_{\mathbf{D}}$ exists with $\mathcal{I}_{\mathbf{D}}(G) = \top$, otherwise $G$ is (simply) \textit{D-unsatisfiable}.
\vspace{2em}

An RDF graph $G$ (simply) \textit{D-entails} an RDF graph $H$, when every D-interpretation which D-satisfies $G$ also D-satisfies $H$.
}


\frame{
\frametitle{More RDF Entailment Regimes}
\framesubtitle{RDF Vocabulary and Axiomatic Triples}
The \textit{RDF Vocabulary} $\text{Voc}_{\text{RDF}}$ is an infinite set  of RDF terms given as follows: 
\tiny
\begin{eqnarray*}
\text{Voc}_{\text{RDF}} = \{ &&\text{\texttt{rdf:type}}, \text{\texttt{rdf:subject}}, \text{\texttt{rdf:predicate}}, \text{\texttt{rdf:object}}, \text{\texttt{rdf:first}}, \text{\texttt{rdf:rest}}, \text{\texttt{rdf:value}}, \\
				       	 &&\text{\texttt{rdf:nil}}, \text{\texttt{rdf:List}}, \text{\texttt{rdf:langString}}, \text{\texttt{rdf:Property}},\text{\texttt{rdf:\_1}}, \text{\texttt{rdf:\_2}}, \dots \}
\end{eqnarray*}
\normalsize

The infinite set of \textit{axiomatic RDF triples} is given as follows:
\tiny
\begin{eqnarray*}
\{ &&( \text{\texttt{rdf:type}}, \text{\texttt{rdf:type}}, \text{\texttt{rdf:Property}} ) , ( \text{\texttt{rdf:subject}}, \text{\texttt{rdf:type}}, \text{\texttt{rdf:Property}} ) ,\\
&&( \text{\texttt{rdf:predicate}}, \text{\texttt{rdf:type}}, \text{\texttt{rdf:Property}} ) , ( \text{\texttt{rdf:object}}, \text{\texttt{rdf:type}}, \text{\texttt{rdf:Property}} ) ,\\
&&( \text{\texttt{rdf:first}}, \text{\texttt{rdf:type}}, \text{\texttt{rdf:Property}} ) , ( \text{\texttt{rdf:rest}}, \text{\texttt{rdf:type}}, \text{\texttt{rdf:Property}} ) ,\\
&&( \text{\texttt{rdf:value}}, \text{\texttt{rdf:type}}, \text{\texttt{rdf:Property}} ) , ( \text{\texttt{rdf:nil}}, \text{\texttt{rdf:type}}, \text{\texttt{rdf:List}} ) ,\\
&&( \text{\texttt{rdf:\_1}}, \text{\texttt{rdf:type}}, \text{\texttt{rdf:Property}} ) , ( \text{\texttt{rdf:\_2}}, \text{\texttt{rdf:type}}, \text{\texttt{rdf:Property}} ) , \dots \}
\end{eqnarray*}
\normalsize
}

\frame{
\frametitle{More RDF Entailment Regimes}
\framesubtitle{RDF-Interpretation, RDF-Satisfiability and RDF-Entailment}
\small
An \textit{RDF-interpretation recognising D} is a (simple) D-interpretation $\mathcal{I}_{\mathbf{D}}$ with $\text{\texttt{xsd:string}} \in \mathbf{D}$ and $\text{\texttt{rdf:langString}} \in \mathbf{D}$ satisfying the following conditions
\begin{itemize}
\item $( x, \mathcal{I}_{\mathbf{D}}(\text{\texttt{rdf:Property}}) ) \in {\circ}_\mathbf{Ext}(\mathcal{I}_{\mathbf{D}}(\text{\texttt{rdf:type}})) \iff x \in \mathbb{P}$
\item $\forall \text{ddd} \in \mathbf{D}: ( x, \mathcal{I}_{\mathbf{D}}(\text{ddd}) ) \in {\circ}_\mathbf{Ext}(\mathcal{I}_{\mathbf{D}}(\text{\texttt{rdf:type}})) \iff x \in \mathcal{V}_{\mathcal{I}_{\mathbf{D}}(\text{ddd})}$
\end{itemize}
as well as every RDF triple in the infinite set of \textit{RDF axioms}.
\vspace{2em}

An RDF graph $G$ is \textit{RDF-satisfiable} (recognising $\mathbf{D}$) if there exists an RDF-interpretation (recognising $\mathbf{D}$) with $\mathcal{I}_{\mathbf{D}}(G) = \top$, otherwise $G$ is \textit{RDF-unsatisfiable}.
\vspace{2em}

An RDF graph $G$ \textit{RDF-entails} an RDF graph $H$ (recognising $\mathbf{D}$), when every RDF-interpretation (recognising $\mathbf{D}$) which RDF-satisfies $G$ also RDF-satisfies $H$.
\normalsize
}



\frame{
\frametitle{More RDF Entailment Regimes}
\framesubtitle{Overview}

\begin{centering}
\begin{figure}
\includegraphics[width=0.9\linewidth]{entailment_regimes.pdf}
\end{figure}        
\end{centering}

}


% ========================================================================================
\section{Graph Morphisms and RDF}
\frame{\sectionpage}

\frame{
\frametitle{Graph Morphisms}
\framesubtitle{Overview}
\begin{center}
\begin{tabular}{ccc}
Homomorphism & $\Leftarrow$ & Isomorphism \\
$\Uparrow$ & & $\Uparrow$ \\
Endomorphism & $\Leftarrow$ & Automorphism
\end{tabular}
\end{center}
}


\frame{
\frametitle{Graph Homomorphism}
\framesubtitle{Definition}
\begin{center}
\begin{tabular}{ccc}
\textbf{Homomorphism} & $\Leftarrow$ & Isomorphism \\
$\Uparrow$ & & $\Uparrow$ \\
Endomorphism & $\Leftarrow$ & Automorphism
\end{tabular}
\end{center}

\vspace{2em}

Given two undirected graphs $G = (V_G, E_G)$ and $H = (V_H, E_H)$, a mapping $\beta : V_G \rightarrow V_H$ is a \textit{homomorphism} from $G$ to $H$ if and only if 
\begin{equation*}
(u, v) \in E_G \implies (\beta(u), \beta(v)) \in E_H
\end{equation*}
}


\frame{
\frametitle{Graph Isomorphism}
\framesubtitle{Definition}
\begin{center}
\begin{tabular}{ccc}
Homomorphism & $\Leftarrow$ & \textbf{Isomorphism} \\
$\Uparrow$ & & $\Uparrow$ \\
Endomorphism & $\Leftarrow$ & Automorphism
\end{tabular}
\end{center}

\vspace{2em}

Given two undirected graphs $G = (V_G, E_G)$ and $H = (V_H, E_H)$, a bijection $\beta : V_G \rightarrow V_H$ is an \textit{isomorphism} from $G$ to $H$ if and only if
\begin{equation*}
(u,v) \in E_G \Leftrightarrow (\beta(u), \beta(v)) \in E_H
\end{equation*}
}

\frame{
\frametitle{Graph Endomorphism}
\framesubtitle{Definition}
\begin{center}
\begin{tabular}{ccc}
Homomorphism & $\Leftarrow$ & Isomorphism \\
$\Uparrow$ & & $\Uparrow$ \\
\textbf{Endomorphism} & $\Leftarrow$ & Automorphism
\end{tabular}
\end{center}

\vspace{2em}

Given an undirected graph $G = (V_G, E_G)$, a mapping $\beta : V_G \rightarrow V_G$ is an \textit{endomorphism} if and only if it is an homomorphism from $G$ to $G$.
}

\frame{
\frametitle{Graph Automorphism}
\framesubtitle{Definition}
\begin{center}
\begin{tabular}{ccc}
Homomorphism & $\Leftarrow$ & Isomorphism \\
$\Uparrow$ & & $\Uparrow$ \\
Endomorphism & $\Leftarrow$ & \textbf{Automorphism}
\end{tabular}
\end{center}

\vspace{2em}

Given an undirected graph $G = (V_G, E_G)$, a mapping $\beta : V_G \rightarrow V_G$ is an \textit{automorphism} if and only if it is an isomorphism from $G$ to $G$.
}

\frame{
\frametitle{Morphisms on RDF Graphs}
\framesubtitle{Blank node mappings and RDF homomorphisms}

\begin{center}
\begin{tabular}{ccc}
\textbf{Homomorphism} & $\Leftarrow$ & Isomorphism \\
$\Uparrow$ & & $\Uparrow$ \\
Endomorphism & $\Leftarrow$ & Automorphism
\end{tabular}
\end{center}

\vspace{2em}

Let $\mu: \mathbf{ILB} \rightarrow \mathbf{ILB}$ be a partial mapping of RDF terms to RDF terms.
If $\mu$ is the identity on $\mathbf{IL}$, we call it a \textit{blank node mapping}. 

\vspace{2em}

Two RDF graphs $G$ and $H$ are \textit{homomorphic} if and only if there exists a blank node mapping $\mu$ such that $\mu(H) \subseteq G$.
}

\frame{
\frametitle{Morphisms on RDF Graphs}
\framesubtitle{RDF isomorphism}

\begin{center}
\begin{tabular}{ccc}
Homomorphism & $\Leftarrow$ & \textbf{Isomorphism} \\
$\Uparrow$ & & $\Uparrow$ \\
Endomorphism & $\Leftarrow$ & Automorphism
\end{tabular}
\end{center}

\vspace{2em}

Two RDF graphs $G$ and $H$ are \textit{isomorphic}, denoted $G \cong H$, if and only if there exists a \textit{bijective} blank node mapping $\mu$ such that $\mu(G) = H$, in which case we call $\mu$ an \textit{isomorphism}.
}

\frame{
\frametitle{Morphisms on RDF Graphs}
\framesubtitle{RDF automorphism}

\begin{center}
\begin{tabular}{ccc}
Homomorphism & $\Leftarrow$ & Isomorphism \\
$\Uparrow$ & & $\Uparrow$ \\
Endomorphism & $\Leftarrow$ & \textbf{Automorphism}
\end{tabular}
\end{center}

\vspace{2em}

A blank node mapping $\mu$ is an \textit{automorphism} of an RDF graph $G$ if $\mu(G) = G$. 
If $\mu$ is the identity mapping on blank nodes in $G$ then $\mu$ is a \textit{trivial automorphism}; otherwise $\mu$ is a \textit{non-trivial automorphism}. 
We denote the set of all automorphisms of $G$ by $Aut(G)$.
}


\frame{
\frametitle{Morphisms on RDF Graphs}
\framesubtitle{RDF endomorphism}

\begin{center}
\begin{tabular}{ccc}
Homomorphism & $\Leftarrow$ & Isomorphism \\
$\Uparrow$ & & $\Uparrow$ \\
\textbf{Endomorphism} & $\Leftarrow$ & Automorphism
\end{tabular}
\end{center}

\vspace{2em}

Given an RDF graph $G$, we denote by $End(G)$ all blank node mappings with domain $terms(G)$ that map $G$ to a subgraph of itself: 
\begin{equation*}
End(G) = \{ \mu \; | \; \mu(G) \subseteq G \land dom(\mu) = terms(G) \}
\end{equation*}
If $\mu(G) \subset G$, we call it a proper endomorphism.
}


\frame{
\frametitle{Morphisms on RDF Graphs}
\framesubtitle{RDF core endomorphism}

\begin{center}
\begin{tabular}{ccc}
Homomorphism & $\Leftarrow$ & Isomorphism \\
$\Uparrow$ & & $\Uparrow$ \\
\textbf{Endomorphism} & $\Leftarrow$ & Automorphism
\end{tabular}
\end{center}

\vspace{2em}

We denote by $CEnd(G) \subset End(G)$ the set of all mappings that map to the fewest unique blank nodes
\footnotesize\begin{equation*}
CEnd(G) = \{  \mu \in End(G) \; | \; \nexists \mu' \in End(G): | codom(\mu' ) \cap \mathbf{B} | < | codom(\mu) \cap \mathbf{B} | \}
\end{equation*}
\normalsize
We call $CEnd(G)$ the \textit{core endomorphisms} of $G$.
}


% ========================================================================================
\section{RDF Morphisms and s-Entailment}
\frame{\sectionpage}

\frame{
\frametitle{RDF Morphisms and s-Entailment}
\framesubtitle{s-Entailment}
We say that the RDF graph $G$ \textit{s-entails} the RDF graph $H$ ($G \models H$) if and only if every model $\mathcal{I}$ which satisfies $G$ also satisfies $H$. 
\vspace{2em}

An RDF graph $G$ \textit{s-entails} an RDF graph $H$ ($G \models H$) if and only if $\exists \mu \in Hom(H) : \mu(H) \subseteq G$.
}

\frame{
\frametitle{RDF Morphisms and s-Entailment}
\framesubtitle{s-Equivalence}
An RDF graph $G$ is \textit{s-equivalent} with an RDF graph $H$ if it holds that $G$ and $H$ are isomorphic, i.e. $(G \cong H) \Rightarrow (G \equiv H)$.
However, $(G \equiv H) \nRightarrow (G \cong H)$!
\vspace{2em}

An RDF graph $G$ is \textit{s-equivalent} with an RDF graph $H$ if and only if $\exists \mu \in CEnd(G), \mu' \in CEnd(H): \mu(G) \cong \mu'(H)$.
}


% ========================================================================================
\section{Morphisms, s-Entailment and Query Answering}
\frame{\sectionpage}

\frame{
\frametitle{Morphisms, s-Entailment and Query Answering}
\framesubtitle{Decision and Evaluation Problem}

Given a query graph $q \in \mathbf{IB}  \times \mathbf{IB} \times \mathbf{IBL}$ and an RDF graph $G \subset \mathbf{IB} \times \mathbf{I} \times \mathbf{IBL}$.
\vspace{2em}

We say that $G$ \textit{answers} $q$ if and only if $G \models q$.
\vspace{2em}

The set $\{ \mu: \mathbf{IBL} \rightarrow \mathbf{ILB} \; | \; \mu = id_{\mathbf{IL}} \; \land \; \exists G' \subseteq G : \mu(q) \cong G' \}$ is the set of all solution mappings of $q$ with respect to $G$. 
}



% ========================================================================================

\section{SPARQL Protocol and RDF Query Language}	
\frame{\sectionpage}

\frame{
\frametitle{SPARQL Protocol and RDF Query Language}
\framesubtitle{SPARQL graph pattern expressions: Syntax}

Let $\mathbf{V}$ denote the infinite set of variables that is disjoint from $\mathbf{IBL}$.\\
\textit{SPARQL graph pattern} expressions are defined recursively as:
\vspace{2em}

\tiny
\begin{enumerate}[(1)]
\item	A triple pattern $t \in \mathbf{IBV}  \times \mathbf{IBV} \times \mathbf{IBLV}$ is a graph pattern.
\item	If $P_1$ and $P_2$ are graph patterns, then expression $(P_{1} \; \text{\texttt{AND}} \; P_2)$, $(P_{1} \; \text{\texttt{UNION}} \; P_2)$, $(P_{1} \; \text{\texttt{OPT}} \; P_2)$ and $(P_{1} \; \text{\texttt{MINUS}} \; P_2)$ are graph patterns.
%\item	If $P_1$ and $P_2$ are graph patterns, then the expression $(P_{1} \; \text{\texttt{UNION}} \; P_2)$ is a (\textit{disjunctive}) graph pattern.
%\item	If $P_1$ and $P_2$ are graph patterns, then the expression $(P_{1} \; \text{\texttt{OPT}} \; P_2)$ is a (\textit{optional}) graph pattern.
\item If $P$ is a graph pattern and $R$ is a SPARQL built-in condition, then the expression $(P \; \text{\texttt{FILTER}} \; R)$ is a graph pattern.
\item If $P$ is a graph pattern and $a \in \mathbf{IV}$, then the expression  $(a \; \text{\texttt{GRAPH}} \; P)$ is a graph pattern.
\item If $P$ is a graph pattern and $a \in \mathbf{IV}$, then the expression  $(a \; \text{\texttt{SERVICE}} \; P)$ is a graph pattern.
\end{enumerate}
\normalsize
}

\frame{
\frametitle{SPARQL Protocol and RDF Query Language}
\framesubtitle{SPARQL graph pattern expressions: Semantics}
The semantics of SPARQL graph patterns is defined in terms of an evaluation function $\llbracket \cdot \rrbracket_{G}^{\mathcal{D}}$.
\vspace{2em}

$\llbracket \cdot \rrbracket_{G}^{\mathcal{D}}$ returns a \textit{set of mappings} for a given SPARQL graph pattern expression, a fixed and \textit{active dataset} $\mathcal{D}$ and an \textit{active graph} $G$ within $\mathcal{D}$.
\vspace{2em}

A \textit{mapping} is a partial function $\mu: \mathbf{V} \rightarrow \mathbf{IBL}$.
}

\frame{
\frametitle{SPARQL Protocol and RDF Query Language}
\framesubtitle{Mappings}


Two mappings $\mu_1$ and $\mu_2$ are \textit{compatible} (written as $\mu_1 \sim \mu_2$) if $\forall v \in dom(\mu_1) \cap dom(\mu_2): \mu_1(v) = \mu_2(v)$.
\vspace{2em}

The mapping with empty domain, denoted by $\mu_{\varnothing}$, is compatible with every other mapping.
\vspace{2em}
If $\mu_1 \sim \mu_2$, then we write $\mu_1 \cup \mu_2$ for the mapping obtained by \textit{extending} $\mu_1$ according to $\mu_2$ on all the variables in $dom(\mu_2) \setminus dom(\mu_1)$.

}

\frame{
\frametitle{SPARQL Protocol and RDF Query Language}
\framesubtitle{Mappings}
Given two sets of mappings $\Omega_1$ and $\Omega_2$, we define

\begin{enumerate}[(1)]
\setlength\itemsep{1em}
 	\item $(\Omega_1 \Join \Omega_2) = \{ \mu_1 \cup \mu_2 \; | \; \mu_1 \in \Omega_1, \; \mu_2 \in \Omega_2, \; \mu_1 \sim \mu_2 \}$
	\item $(\Omega_1 \cup \Omega_2) = \{ \mu \; | \; \mu \in (\Omega_1 \cup \Omega_2) \}$
	\item $(\Omega_1 \setminus \Omega_2) = \{ \mu \in \Omega_1 \; | \; \forall \mu' \in \Omega_2 : \mu \not\sim \mu' \}$
	\item $(\Omega_1 \leftouterjoin \Omega_2) = (\Omega_1 \Join \Omega_2 ) \cup (\Omega_1 \setminus \Omega_2)$
	\item $\pi_V (\Omega) = \{ \mu' \; | \; \exists \mu \in \Omega: \mu' \subseteq \mu, dom(\mu') = V \cap dom(\mu) \}$
 \end{enumerate}
}

\begin{frame}[allowframebreaks]
\frametitle{SPARQL Protocol and RDF Query Language}
\framesubtitle{SPARQL graph pattern expressions: Semantics}
Evaluation $\llbracket P \rrbracket_{G}^{\mathcal{D}}$ of a SPARQL graph pattern $P$ over a dataset $\mathcal{D}$ with active graph $G$ is defined recursively as
\vspace{2em}

\footnotesize
\begin{enumerate}[(1)]
\setlength\itemsep{1em}
	\item	If $P$ is a triple pattern, then  $\llbracket P \rrbracket_{G}^{\mathcal{D}} = \{ \mu: var(P) \to \mathbf{IBL} \; | \; \mu(P) \in G \}$.
	\item If $P = ( P_1 \; \text{\texttt{AND}} \; P_2 )$, then $\llbracket P \rrbracket_{G}^{\mathcal{D}} = \llbracket P_1 \rrbracket_G \Join \llbracket P_2 \rrbracket_G$.
	\item If $P = ( P_1 \; \text{\texttt{UNION}} \; P_2 )$, then $\llbracket P \rrbracket_{G}^{\mathcal{D}} = \llbracket P_1 \rrbracket_G \cup \llbracket P_2 	\rrbracket_G$.
	\item If $P = ( P_1 \; \text{\texttt{OPT}} \; P_2 )$, then $\llbracket P \rrbracket_{G}^{\mathcal{D}} = \llbracket P_1 \rrbracket_G \leftouterjoin \llbracket P_2 \rrbracket_G$.
	\item If $P = ( P_1 \; \text{\texttt{MINUS}} \; P_2 )$, then $\llbracket P \rrbracket_{G}^{\mathcal{D}} = \llbracket P_1 \rrbracket_G \setminus \llbracket P_2 \rrbracket_G$.
	\item If $P = ( P_1 \; \text{\texttt{FILTER}} \; R )$, then $\llbracket P \rrbracket_G = \{ \mu \in \llbracket P_1 \rrbracket _G \; | \; \mu \models R \}$.
	\item If $P = (g \; \text{\texttt{GRAPH}} \; P_1)$ with $g \in \mathbf{IV}$, then
	\begin{equation*}
	\llbracket P \rrbracket_{G}^{\mathcal{D}} = \begin{cases}
	\llbracket P_1 \rrbracket^{\mathcal{D}}_{\sigma_{\mathcal{D}}\lbrack g \rbrack} \Leftrightarrow g \in \nu_{\mathcal{D}}  \\
%\{ \mu_{\varnothing} \} \Leftrightarrow g \not\in \nu_{\mathcal{D}}\\
	\bigcup_{u \in \mathbf{I}}(\llbracket P_1 \rrbracket^{\mathcal{D}}_{\sigma_{\mathcal{D}}\lbrack u \rbrack} \Join \{ \lbrack g \mapsto u \rbrack \} ) \Leftrightarrow g \in \mathbf{V}
%\{ \mu_\varnothing \} \Leftrightarrow a \in \mathbf{I} , \sigma_{\mathcal{D}}\lbrack a \rbrack = \varnothing \\
%\{ \mu \cup \mu_{a} \; | \; \exists g \in \nu_{\mathcal{D}}: \mu_{a} = \lbrack a \rightarrow g \rbrack, \mu \in \llbracket P_1 \rrbracket^{\sigma_{\mathcal{D}}\lbrack g \rbrack} \land \mu_c \sim \mu \} \; \text{iff} \; a \in \mathbf{V}\\
%\{ \mu \cup \lbrack a \mapsto g \rbrack \; | \; \exists g \in \nu_{\mathcal{D}}: \mu \in \llbracket P_1 \rrbracket^{\sigma_{\mathcal{D}}\lbrack g \rbrack} , \mu \sim \lbrack a \mapsto g \rbrack  \} \Leftrightarrow a \in \mathbf{V}
	\end{cases}
	\end{equation*}
	\item If $P = (s \; \text{\texttt{SERVICE}} \; P_1)$ with $s \in \mathbf{IV}$, then
	\begin{equation*}
		\llbracket P \rrbracket_{G}^{\mathcal{D}} = \begin{cases}
		\llbracket P_1 \rrbracket^{\mathcal{D}}_{\sigma_{\mathcal{D}}\lbrack s \rbrack} \Leftrightarrow s \in \mathbf{I}  \\
		\{ \mu_{\varnothing} \} \Leftrightarrow s \not\in \nu_{\mathcal{D}}\\ 
		TBD
		\end{cases}
	\end{equation*}
\end{enumerate}
\normalsize
\end{frame}

%\frame{
%\frametitle{SPARQL Protocol and RDF Query Language}
%\framesubtitle{SPARQL graph pattern expressions: Semantics}
%\footnotesize
%\begin{enumerate}[(1)]
%\setlength\itemsep{1em}
%\setcounter{enumi}{5}
%%\item If $P = ( P_1 \; \text{\texttt{FILTER}} \; R )$, then $\llbracket P \rrbracket_G = \{ \mu \in \llbracket P_1 \rrbracket _G \; | \; \mu \models R \}$..
%\item If $P = (g \; \text{\texttt{GRAPH}} \; P_1)$ with $g \in \mathbf{IV}$, then
%\begin{equation*}
%\llbracket P \rrbracket_{G}^{\mathcal{D}} = \begin{cases}
%\llbracket P_1 \rrbracket^{\mathcal{D}}_{\sigma_{\mathcal{D}}\lbrack g \rbrack} \Leftrightarrow g \in \nu_{\mathcal{D}}  \\
%%\{ \mu_{\varnothing} \} \Leftrightarrow g \not\in \nu_{\mathcal{D}}\\
%\bigcup_{u \in \mathbf{I}}(\llbracket P_1 \rrbracket^{\mathcal{D}}_{\sigma_{\mathcal{D}}\lbrack u \rbrack} \Join \{ \lbrack g \mapsto u \rbrack \} ) \Leftrightarrow g \in \mathbf{V}
%%\{ \mu_\varnothing \} \Leftrightarrow a \in \mathbf{I} , \sigma_{\mathcal{D}}\lbrack a \rbrack = \varnothing \\
%%\{ \mu \cup \mu_{a} \; | \; \exists g \in \nu_{\mathcal{D}}: \mu_{a} = \lbrack a \rightarrow g \rbrack, \mu \in \llbracket P_1 \rrbracket^{\sigma_{\mathcal{D}}\lbrack g \rbrack} \land \mu_c \sim \mu \} \; \text{iff} \; a \in \mathbf{V}\\
%%\{ \mu \cup \lbrack a \mapsto g \rbrack \; | \; \exists g \in \nu_{\mathcal{D}}: \mu \in \llbracket P_1 \rrbracket^{\sigma_{\mathcal{D}}\lbrack g \rbrack} , \mu \sim \lbrack a \mapsto g \rbrack  \} \Leftrightarrow a \in \mathbf{V}
%\end{cases}
%\end{equation*}
%\item If $P = (s \; \text{\texttt{SERVICE}} \; P_1)$ with $s \in \mathbf{IV}$, then
%\begin{equation*}
%\llbracket P \rrbracket_{G}^{\mathcal{D}} = \begin{cases}
%\llbracket P_1 \rrbracket^{\mathcal{D}}_{\sigma_{\mathcal{D}}\lbrack s \rbrack} \Leftrightarrow s \in \mathbf{I}  \\
%\{ \mu_{\varnothing} \} \Leftrightarrow s \not\in \nu_{\mathcal{D}}\\ 
%TBD
%\end{cases}
%\end{equation*}
%\end{enumerate}
%\normalsize
%}

\frame{
\frametitle{SPARQL Protocol and RDF Query Language}
\framesubtitle{SPARQL \texttt{SELECT} Queries}
A \texttt{SELECT} query $\mathbf{q}$ is an expression of the form 
\begin{equation*}
\mathbf{q} = \text{\texttt{SELECT}}_{V} \; \text{\texttt{WHERE}} \; \{ \; P \; \}
\end{equation*}
where $V \subseteq var(P)$, and $P$ is a SPARQL graph pattern.
\vspace{2em}

The semantics of a  \texttt{SELECT} query over an input dataset $\mathcal{D}$ with active graph $G$ is given as 
\begin{equation*}
ans(\mathbf{q},\mathcal{D}) = \pi_V ( \llbracket P \rrbracket^{\mathcal{D}}_{G}) 
\end{equation*}
}

\begin{frame}[fragile]
\frametitle{SPARQL Protocol and RDF Query Language}
\framesubtitle{Example}

RDF Graph
\begin{center}
\begin{tabular}{c}
\begin{lstlisting}[language=ttl,basicstyle=\tiny]
@prefix foaf:  <http://xmlns.com/foaf/0.1/> .

_:a  foaf:name   "Johnny Lee Outlaw" .
_:a  foaf:mbox   <mailto:jlow@example.com> .
_:b  foaf:name   "Peter Goodguy" .
_:b  foaf:mbox   <mailto:peter@example.org> .
_:c  foaf:mbox   <mailto:carol@example.org> .
\end{lstlisting}
\end{tabular}
\end{center}

SPARQL Query
\begin{center}
\begin{tabular}{c}
\begin{lstlisting}[language=sparql,basicstyle=\tiny]
PREFIX foaf: <http://xmlns.com/foaf/0.1/>

SELECT ?name ?mbox
WHERE { ?x foaf:name ?name ; foaf:mbox ?mbox . }
\end{lstlisting}
\end{tabular}
\end{center}

Result
\begin{center}
\begin{tiny}
\begin{tabular}{ | c | c | }
\hline
name & mbox \\
\hline
"Johnny Lee Outlaw"	&  mailto:jlow@example.com \\
\hline
"Peter Goodguy" & mailto:peter@example.org \\
\hline
\end{tabular}
\end{tiny}
\end{center}
\end{frame}


\frame{
\frametitle{SPARQL Protocol and RDF Query Language}
\framesubtitle{SPARQL \texttt{CONSTRUCT} Queries}
A \texttt{CONSTRUCT} query $\mathbf{q}$ is an expression of the form 
\begin{equation*}
\mathbf{q} = \text{\texttt{CONSTRUCT}} \; H \; \text{\texttt{WHERE}} \; \{ \; P \; \}
\end{equation*}
where $H \subset \mathbf{IBV} \times \mathbf{IBV} \times \mathbf{IBLV}$ is a finite set of triple patterns, called a template, and $P$ is a SPARQL graph pattern.
\vspace{2em}

The semantics of a  \texttt{CONSTRUCT} query over an input dataset $\mathcal{D}$ with active graph $G$ is given as 
\begin{equation*}
ans(\mathbf{q},\mathcal{D}) = \{ \mu(t) \| \mu \in \llbracket P \rrbracket^{\mathcal{D}}_{G}, t \in H \}
\end{equation*}
}

\begin{frame}[fragile]
\frametitle{SPARQL Protocol and RDF Query Language}
\framesubtitle{Example}

RDF Graph
\begin{center}
\begin{tabular}{c}
\begin{lstlisting}[language=ttl,basicstyle=\tiny]
@prefix org:    <http://example.com/ns#> .

_:a  org:employeeName   "Alice" .
_:a  org:employeeId     12345 .
_:b  org:employeeName   "Bob" .
_:b  org:employeeId     67890 .
\end{lstlisting}
\end{tabular}
\end{center}

SPARQL Query
\begin{center}
\begin{tabular}{c}
\begin{lstlisting}[language=sparql,basicstyle=\tiny]
PREFIX foaf:   <http://xmlns.com/foaf/0.1/>
PREFIX org:    <http://example.com/ns#>

CONSTRUCT { ?x foaf:name ?name }
WHERE  { ?x org:employeeName ?name }
\end{lstlisting}
\end{tabular}
\end{center}

Result
\begin{center}
\begin{tabular}{c}
\begin{lstlisting}[language=ttl,basicstyle=\tiny]
@prefix org: <http://example.com/ns#> .
      
_:x foaf:name "Alice" .
_:y foaf:name "Bob" .\end{lstlisting}
\end{tabular}
\end{center}
\end{frame}


% ========================================================================================
\section{SPARQL Query Containment and Equivalence}
\frame{\sectionpage}

\frame{
\frametitle{SPARQL Query Containment}
\framesubtitle{Definition}
Qu�ery containment is the problem of deciding whether the result set of one query is included in the result set of another.
\vspace{2em}

Given two queries $q_1$ and $q_2$, we say that $q_2$ \textit{contains} $q_1$ iff
\begin{equation*}
q_1 \sqsubseteq q_2 \Leftrightarrow \forall G\subset \mathcal{T}: ans(\mathbf{q_1}, \{ G \}) \subseteq ans(\mathbf{q_2}, \{ G \})
\end{equation*}
\vspace{2em}

Note that this definition does \textit{not} depend on a specific RDF graph or RDF dataset!
}

\begin{frame}[fragile]
\frametitle{SPARQL Query Containment}
\framesubtitle{Example}

SPARQL Query $q_1$
\begin{center}
\begin{tabular}{c}
\begin{lstlisting}[language=sparql,basicstyle=\tiny]
PREFIX ex: <http://example.com/ns#>

SELECT ?x ?y ?z 
WHERE { 
  ?x ex:sister ?y . 
  ?y ex:name ?z . 
}


\end{lstlisting}
\end{tabular}
\end{center}

SPARQL Query $q_2$
\begin{center}
\begin{tabular}{c}
\begin{lstlisting}[language=sparql,basicstyle=\tiny]
SELECT *
WHERE { ?s ?p ?o. }
\end{lstlisting}
\end{tabular}
\end{center}

\end{frame}

\frame{
\frametitle{SPARQL Query Equivalence}
\framesubtitle{Definition}
Qu�ery equivalence is the problem of deciding whether the result set of one query is equivalent to the result set of another.
\vspace{2em}

Given two queries $q_1$ and $q_2$, we say that $q_1$ and $q_2$ are \textit{equivalent} iff
\begin{equation*}
q_1 \equiv q_2 \Leftrightarrow q_1 \sqsubseteq q_2 \land q_2 \sqsubseteq q_1
\end{equation*}
\vspace{2em}

Note that definition of query containment and equivalence does \textit{not} depend on a specific RDF graph or RDF dataset!
}

\begin{frame}[fragile]
\frametitle{SPARQL Query Equivalence}
\framesubtitle{Example}

SPARQL Query $q_1$
\begin{center}
\begin{tabular}{c}
\begin{lstlisting}[language=sparql,basicstyle=\tiny]
PREFIX ex: <http://example.com/ns#>

SELECT ?z 
WHERE {
  { ?x ex:sister ?y . 
    ?y ex:name ?z . }
  { ?w ex:mother ?x . }
  UNION 
  { ?w ex:father ?x. } 
}

\end{lstlisting}
\end{tabular}
\end{center}

SPARQL Query $q_2$
\begin{center}
\begin{tabular}{c}
\begin{lstlisting}[language=sparql,basicstyle=\tiny]
PREFIX ex: <http://example.com/ns#>

SELECT ?n 
WHERE {
  { ?s :name ?n . 
    ?c :mother ?p . 
    ?p :sister ?s . }
  UNION
  { ?s :name ?n . 
    ?c :father ?p . 
    ?p :sister ?s . } 
}
\end{lstlisting}
\end{tabular}
\end{center}
\end{frame}

\frame{
\frametitle{SPARQL Query Containment \& Equivalence}
\framesubtitle{Motivation}
\textit{Rewriting}: find a \textit{minimal} query equivalent to the original one 
\vspace{2em}

\textit{Optimization}: reduce the computational cost of query evaluation using minimized queries
\vspace{2em}

\textit{Caching}:  increase throughput of query processing by re-using work done for (containing) previous queries
\vspace{2em}

\textit{Composition}: perform (partial) composition of \texttt{CONSTRUCT} queries $q_1$ and $q_2$ where $P_{q_2} \sqsubseteq H_{q_1}$
}


\begin{frame}[allowframebreaks]
\frametitle{SPARQL Query Containment \& Equivalence}
\framesubtitle{Related Work}
\tiny
\nocite{*}
\bibliographystyle{apalike}
\bibliography{Entailment.bib,Fragments.bib,Benchmarks.bib}
\normalsize
\end{frame}
%
%\begin{frame}
%\frametitle{SPARQL Query Containment \& Equivalence}
%\framesubtitle{Related Work wrt. SPARQL Fragments}
%\tiny
%%\nocite{*}
%%\bibliographystyle{apalike}
%%\bibliography{Fragments.bib}
%\normalsize
%\end{frame}
%
%\begin{frame}
%\frametitle{SPARQL Query Containment \& Equivalence}
%\framesubtitle{Related Work wrt. Benchmarks}
%\tiny
%%\nocite{*}
%%\bibliographystyle{apalike}
%%\printbibliography{Benchmarks} 
%\normalsize
%\end{frame}


% ========================================================================================
\section{RDF Morphisms and SPARQL Query Containment}
\frame{\sectionpage}


\begin{frame}[fragile]
\frametitle{RDF Morphisms and SPARQL Query Containment}
\framesubtitle{Representing SPARQL Queries as RDF Graphs}

\begin{lstlisting}[language=sparql,basicstyle=\tiny]
PREFIX : <http://dfki.de/voc#>

SELECT ?z 
WHERE { 
  { ?x :sister ?y .
    ?y :name ?z . }
  UNION
  { ?a ?s ?b .
    ?m ?n ?z . } }
\end{lstlisting}

\begin{centering}
\begin{figure}
\includegraphics[width=0.6\linewidth]{r_graph_ex.pdf}
\end{figure}        
\end{centering}
\end{frame}


\begin{frame}[fragile]
\frametitle{RDF Morphisms and SPARQL Query Containment}
\framesubtitle{Algebraic Transformation into Union Normal Form}

\begin{centering}
\begin{figure}
\includegraphics[width=0.6\linewidth]{r_graph_ex.pdf}
\end{figure}        
\end{centering}
\end{frame}


\begin{frame}[fragile]
\frametitle{RDF Morphisms and SPARQL Query Containment}
\framesubtitle{Computation of RDF cores for intra-dependencies removal in sub-CQs}

\begin{centering}
\begin{figure}
\includegraphics[width=0.35\linewidth]{cq_1.pdf}
\end{figure}        
\end{centering}

\begin{centering}
\begin{figure}
\includegraphics[width=0.35\linewidth]{cq_2.pdf}
\end{figure}        
\end{centering}
\end{frame}


\begin{frame}[fragile]
\frametitle{RDF Morphisms and SPARQL Query Containment}
\framesubtitle{Computation of RDF cores for intra-dependencies removal in sub-CQs}

\begin{centering}
\begin{figure}
\includegraphics[width=0.35\linewidth]{cq_1.pdf}
\end{figure}        
\end{centering}

\begin{centering}
\begin{figure}
\includegraphics[width=0.15\linewidth]{cq_2_min.pdf}
\end{figure}        
\end{centering}
\end{frame}


\begin{frame}[fragile]
\frametitle{RDF Morphisms and SPARQL Query Containment}
\framesubtitle{Computation of subgraph isomorphisms for inter-dependencies removal in sub-CQs}

\begin{centering}
\begin{figure}
\includegraphics[width=0.6\linewidth]{inter_cq.pdf}
\end{figure}        
\end{centering}
\end{frame}

\begin{frame}[fragile]
\frametitle{RDF Morphisms and SPARQL Query Containment}
\framesubtitle{Computation of subgraph isomorphisms for inter-dependencies removal in sub-CQs}

\begin{lstlisting}[language=sparql,basicstyle=\tiny]
PREFIX : <http://dfki.de/voc#>

SELECT ?z 
WHERE { 
  { ?x :sister ?y .
    ?y :name ?z . }
  UNION
  { ?a ?s ?b .
    ?m ?n ?z . } }
\end{lstlisting}

\begin{centering}
\begin{figure}
\includegraphics[width=0.25\linewidth]{query_ex_min}
\end{figure}        
\end{centering}

\end{frame}




\end{document}